\begin{frame}{Выбор биометрических параметров}
	Сигнал ЭЭГ:
	\begin{gather*}
		\overline{s} = \{ s_i(t) \},\; 1 \leq i \leq I
	\end{gather*}
	Далее к сигналу ЭЭГ применяется БПФ:
	\begin{gather*}
		F(\overline{s}) = \{ F(s_i(t)) \} = \{ \overline{a}_i \}
	\end{gather*}
	И формируются следующие векторы:
	\begin{gather*}
		\overline{a_i} = \{ a_{ij} \}, \\
		a_{ij} = \max_{a_i}|a| \cdot cos(arg \; a) : 10^{\circ} < arg \; a_{ik} < 90^{\circ} \\
		1 < i \leq I,\; 1 \leq j \leq J,\; k=\overline{0,RES}
	\end{gather*}
	$\overline{a_i}$ - вектор биометрических данных, используемый в нейросетевом преобразователе; \\
	$I$ - общее кол-во электродов; $J$ - кол-во коэффициентов; \\
	$RES$ - кол-во отсчетов для дискретизации.
\end{frame}
