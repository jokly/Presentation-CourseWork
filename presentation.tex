\documentclass{beamer}

% set font
\usepackage{fontspec}
\setsansfont{Fira Sans}
\setmonofont{Fira Mono}
\newfontfamily\cyrillicfont{Fira Sans}

% set lang
\usepackage{polyglossia}
\setdefaultlanguage[spelling=modern]{russian}
\setotherlanguage{english}

% set theme
\usetheme{metropolis}

% pakages
\usepackage{caption}

% set info
\title{Биометическая аутентификация с помощью нейроинтерфейса}
\author{Студент: Сластен Т.Д. \\ Руководители: Боршевников А.Е., Кленин А.С.}
\institute{Б8303а Прикладная математика и информатика}
\date{}

% fonts

\begin{document}

\maketitle

\section{Введение}

% 1 slide
\begin{frame}{Технология проведения эксперимента}
	\begin{columns}[T]
  		\begin{column}{.6\textwidth}
     		\begin{block}{Процесс стимуляции}
				Стимуляция выглядит, как поочередно меняющиеся цифры от "0" до "9".
				Пользователь выбирает 1 или 2 символа и при их появлении концентрируется на них.
				Данные символы являются "мысленным" паролем.
    		\end{block}
    	\end{column}
    	\begin{column}{.4\textwidth}
    		\begin{figure}
    			\includegraphics[width=\textwidth]{img/0_4.png}
    			\caption{Фрагмент визуальной стимуляции}
    		\end{figure}
    	\end{column}
  	\end{columns}
\end{frame}

% 2 slide
\begin{frame}{Выбор биометрических параметров}
	К сигналу ЭЭГ применяется быстрое преобразование Фурье.
	Далее отбрасываются коэффициенты, не удовлетворяющие условию:
	\begin{gather*}
		10^{\circ} < arg \; a_i < 90^{\circ}
	\end{gather*}
	И формируются следующие векторы:
	\begin{gather*}
		\overline{a_i} = \{ a_{ij} \}, \\
		a_{ij} = \max_{a_i}|a| \cdot cos(arg \; a),\; 1 < i \leq I,\; 1 \leq j \leq J
	\end{gather*}
	$\overline{a_i}$ - вектор биометрических данных, используемый в нейросетевом преобразователе \\
	$I$ - общее кол-во электродов \\
	$J$ - кол-во выбираемых коэффициентов
\end{frame}

% 3 slide
\begin{frame}{Цель}
	\begin{enumerate}
		\item Изучение требований линейки стандартов ГОСТ Р 52633.0.
		\item Написание нейросетевого преобразователя «Биометрия – код доступа»,
			использующий данные электроэнцефалограммы мозга.
		\item Проведение исследований с преобразователем «Биометрия – код доступа», его оптимизация.
	\end{enumerate}
\end{frame}

\end{document}
